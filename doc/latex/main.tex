\documentclass[14pt,a4paper]{extarticle}
\usepackage{minted,xcolor}
\usemintedstyle{manni}
\definecolor{bg}{HTML}{ffffff}
\usepackage{mathtext}
\usepackage{listings}
\usepackage{color}
\usepackage{float}
\usepackage{graphicx}
\usepackage[T1,T2A]{fontenc}
\usepackage[utf8x]{inputenc}
\usepackage[english,ukrainian]{babel}
\usepackage[a4paper, total={7in, 9.5in}]{geometry}

\usepackage{listings}
\usepackage{lipsum}
\usepackage{hyperref}
\usepackage{longtable}
\usepackage{booktabs}
\usepackage{pdfpages}
\usepackage{fancyhdr}
\usepackage{shading}
\usepackage[ddmmyyyy]{datetime}

\makeatletter
\def\maxwidth#1{\ifdim\Gin@nat@width>#1 #1\else\Gin@nat@width\fi}
\makeatother

\providecommand{\tightlist}{%
  \setlength{\itemsep}{0pt}\setlength{\parskip}{0pt}}

\pagestyle{fancy}
\fancyhf{}
\rhead{Бездротовий зв'язок NRF24L01+}
\lhead{https://github.com/spolischook/ATP}
\fancyfoot[CE,CO]{\rightmark}
\fancyfoot[LE,RO]{\thepage}
\renewcommand{\headrulewidth}{2pt}
\renewcommand{\footrulewidth}{1pt}

\title{Бездротовий зв'язок модулів NRF24L01}
\selectlanguage{ukrainian}
\author{Сергій Поліщук}
\date{\today}
\selectlanguage{english}

% Pandoc syntax highlighting
% See https://github.com/rstudio/rticles/issues/182
\usepackage{fancyvrb}
\newcommand{\passthrough}[1]{\lstset{mathescape=false}#1\lstset{mathescape=true}}
\usepackage{minted}
\begin{document}

\maketitle
\tableofcontents
\clearpage
\section{Вступ}
Наявність двох або більше плат Arduino, здатних спілкуватися між собою
бездротово на відстані, відкриває безліч можливостей, таких як
віддалений моніторинг даних датчиків, управління роботами, домашня
автоматизація та інше. І коли справа зводиться до недорогих, але
надійних двосторонніх радіочастотних рішень, є декілька варіантів:
\begin{itemize}
\item ZeegBee -
\href{https://web.archive.org/web/20100216234546/http://freaklabs.org/index.php/Blog/Zigbee/Zigbee-Linux-and-the-GPL.html}{ліцензований
протокол}, лідер ринку IoT комунікацій
\item nRF24L01 -
\href{https://www.nordicsemi.com/Products/Low-power-short-range-wireless/nRF24-series}{Nordic
Semiconductor}
\item Bluetooth
\item WiFi
\end{itemize}

Модуль трансивера nRF24L01 + (плюс) часто можна отримати в Інтернеті за
\href{https://ru.aliexpress.com/wholesale?SearchText=nrf24l01}{менше
одного долара США}, що робить його одним з найдешевших варіантів радіо
передачі даних, які є сьогодні на ринку. І найкраще, що ці модулі є
надзвичайно крихітними, що дозволяє включати бездротовий інтерфейс майже
в будь-який проект.

Усі приклади робочих проектів виконані за допомогою
\href{https://code.visualstudio.com/}{Visual Studio Code}
та \href{https://platformio.org/}{PlatformIO}  
Документаця написана за допомогою CLion (student license)
разом з Markdown плагіном.  
Таблиці BOM конвертовані онлайн сервісом \href{https://www.convertcsv.com/csv-to-markdown.htm}{ConverCSV}  
Усі схеми розроблено у \href{https://easyeda.com/}{EasyEDA} під ліценцією MIT  
Друкована документація зібрана \href{https://pandoc.org/}{PanDoc}

Внесення змін до документації або коду через систему контролю
версій Git за адресою
\href{https://github.com/spolischook/ATP}{https://github.com/spolischook/ATP}

Останню версію цієї друкованої документації у форматі pdf можна скачати за адресою 
\href{https://github.com/spolischook/ATP/blob/master/doc/latex/main.pdf}{https://github.com/spolischook/ATP/blob/master/doc/latex/main.pdf}

\clearpage

\input{include/theory.tex}
\clearpage
\section{Модулі}

\input{include/modules/07-ArduinoNano-RF24L01-LCD.tex}
\clearpage
\input{include/modules/03-photoresistor.tex}
\clearpage
\input{include/modules/04-termistor.tex}
\clearpage
\input{include/modules/05-ArduinoNano-RF24L01-relay.tex}
\clearpage
\section{Проект комунікації модулів}

\end{document}
